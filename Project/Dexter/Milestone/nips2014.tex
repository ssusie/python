\documentclass{article} % For LaTeX2e
\usepackage{nips14submit_e,times}
\usepackage{hyperref}
\usepackage{url}
%\documentstyle[nips14submit_09,times,art10]{article} % For LaTeX 2.09


\title{Feature Selection for Noisy Data Binary Classification }


\author{
Sargsyan S. \\
Department of Applies Mathematics\\
University of Washington\\
Seattle, WA 98105 \\
\texttt{hippo@cs.cranberry-lemon.edu} \\
\And
Coauthor Syuzanna Sargsyan\\
%Affiliation \\
%
%Address \\
%
\texttt{email}: ssusie@uw.edu \\
\AND
Coauthor Bethany Lusch \\
%Affiliation \\
%
%Address \\
%
\texttt{email} herwaldt@uw.edu  \\
%\And
%
%Coauthor \\
%
%Affiliation \\
%
%Address \\
%
%\texttt{email} \\
%
%\And
%
%Coauthor \\
%
%Affiliation \\
%
%Address \\
%
%\texttt{email} \\
%
%(if needed)\\
%
%}
%

% The \author macro works with any number of authors. There are two commands
% used to separate the names and addresses of multiple authors: \And and \AND.
%
% Using \And between authors leaves it to \LaTeX{} to determine where to break
% the lines. Using \AND forces a linebreak at that point. So, if \LaTeX{}
% puts 3 of 4 authors names on the first line, and the last on the second
% line, try using \AND instead of \And before the third author name.

\newcommand{\fix}{\marginpar{FIX}}
\newcommand{\new}{\marginpar{NEW}}

%\nipsfinalcopy % Uncomment for camera-ready version

\begin{document}


\maketitle

%\begin{abstract}
%
%The abstract paragraph should be indented 1/2~inch (3~picas) on both left and
%
%right-hand margins. Use 10~point type, with a vertical spacing of 11~points.
%
%The word \textbf{Abstract} must be centered, bold, and in point size 12. Two
%
%line spaces precede the abstract. The abstract must be limited to one
%
%paragraph.
%
%\end{abstract}
%

%\section{Section title}
%


%\begin{center}
%
%   \url{http://papers.nips.cc}
%
%\end{center}
%Please read carefully the
%
%instructions below, and follow them faithfully.
%

This work is on binary classification of ``bag-of-words'' type data. The data 
is called Dexter and downloaded from the following website.
\begin{center}
   \url{http://archive.ics.uci.edu/ml/datasets/Dexter}
\end{center}

It is required to classify the Reuter's articles as ``corporate acquisitions'' 
or not. Overall there are 20,000 features out of which only 9947 are real 
features and the rest is a noise. Also 2572 real features are always zeros.
There are several techniques we have tried so far for this classification task.

\section{Support Vector Machine}
Support Vector machine ( \cite{Svm}) is a binary linear classifier. It constructs a hyperplane in a high dimensional space 
to separate two classes. The hyperplane used in the algorithm has the largest 
functional margin which provides a lower generalization error.  We used the built-in Python classifier 
\textit{sklearn.svm.SVC}  and on the validation data we got {\bf 50\%} 
error. This is because the SVM algorithm treats all features as equally important whereas in our data set 
 more then half of the features are irrelevant.

\section{Random Forests}
Random forests algorithm (\cite{RandForests}) uses decision trees for making predictions. It randomly 
chooses subsamples from dataset, constructs decision trees and uses averaging to improve prediction. The 
Python built-in function \textit{sklearn.ensemble.RandomForestClassifier} has 
options for selecting the number of subsamples for constructing the decision 
trees. Without using this option on average we get {\bf 16.26\%} error on the validation set 
by running the classifier 100 times. However, using the fact that we know the number of relevant features is 
 give the number of features is 7375, we get  on average {\bf 11.6\%} error.

\section{Lasso}
Lasso (\cite{Lasso}) is an algorithm for linear regression which ignores some features (sets the coefficient to zero) depending
 on the value of the regularizer. As was said this is a regression algorithm which predicts numerical values but 
 does not do classification. However, as we did in homework 2, we used the algorithm to try to zero out
  irrelevant features then use some threshold for  classification. Again, we used the Python built-in 
function \textit{sklearn.linear\_model.Lasso} and used 0 as a threshold value. 
 we tried the same here an for suitable choice of 
$\lambda$  ($ \lambda= 0.0134217728$), we were able to get {\bf 9.33\%} 
accuracy.\\

\section{Linear Support Vector Classification combined with $L_1$ regularization}
There is an option in the Python  built-in function \textit{sklearn.svm.LinearSVC} (\cite{skilearn})where you can 
require to do a $L_1$ regularization. This basically combines SVM algorithm with 
Lasso. This can be done by setting $penalty='l1'$ and by a suitable choice of 
lambda we were able to get {\bf 7.67\%} accuracy.

\section{Mean of each class}
As we know most of the features in the dataset are useless, therefore we need to do some tricks to get rid 
of them somehow. The idea here was to compute the mean of each feature for two classes separately 
then compare them. If they differ by more than 0.1 the SVM error reduces from $50\%$ 
to $15.67\%$. However there was no essential improvement in other algorithms 
even though we have tried differnet values for separation.


\subsubsection*{Acknowledgments}

We especially thank Charles Delahunt for vary helpful discussions.

\subsubsection*{References}



\small{\begingroup
\renewcommand{\section}[2]{}%

\begin{thebibliography}{5}

 \bibitem{Svm}   Cortes, C.; Vapnik, V. (1995). "Support-vector networks". Machine Learning 20 (3): 273. doi:10.1007/BF00994018
 
 \bibitem{RandForests} Ho, Tin Kam (1995). Random Decision Forests (PDF). Proceedings of the 3rd
  International Conference on Document Analysis and Recognition, Montreal, QC, 14–16 August 1995. pp. 278–282.
 
 \bibitem{Lasso} Tibshirani, R. (1996). Regression shrinkage and selection via the 
 lasso. Journal of the Royal Statistical Society. Series B (Methodological), 
 JSTOR, pp. 267--288
 
 \bibitem{skilearn} Documentation on LinearSVC, 
 \url{http://scikit-learn.org/stable/modules/generated/sklearn.svm.LinearSVC.html}
 
\end{thebibliography}
\endgroup

\end{document}
